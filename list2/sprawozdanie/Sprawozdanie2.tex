\documentclass[titlepage]{article}
\usepackage[utf8]{inputenc}
\usepackage{polski}
\usepackage{amsmath}
\usepackage{graphicx}
\graphicspath{ {./graphics/} }
\title{Sprawozdanie 2}
\date{2019-03-26}
\author{Józef Piechaczek}

\begin{document}

\pagenumbering{gobble}
\maketitle
\newpage
\pagenumbering{arabic}

\section{Zadanie 1}

\subsection{Wstęp}
Celem zadania 1. jest rozpatrzenie niezawodności sieci o modelu $S = <G, H>$ w czasie podzielonym na interwały. 
Podstawowy model składa się z grafu \linebreak $G = <V, E>$ zwierającego 20 wierzchołków oznaczonych przez $v(i)$, dla \linebreak $i = 1,..20$, a zbiór $E$ zawiera 19 krawędzi $e(j,j+1)$, dla $j = 1, ...,19$. Zbiór $H$ zawiera funkcję niezawodności $h$ przyporządkowującą każdej krawędzi $e(j,k)$ ze zbioru $E$ wartość $0.95$ oznaczającą prawdopodobieństwo nieuszkodzenia tego kanału komunikacyjnego w dowolnym interwale. 

\subsection{Symulacja}
W celu oszacowania niezawodności sieci posłużę się programem, korzystającym z metody Monte Carlo, napisanym w języku Java, z pomocą biblioteki JGraphT. W celu wizualizacji grafów posłużę się narzędziem Graphviz. Symulacja będzie polegała na wykonaniu odpowiedniej liczby prób na danym grafie i porównaniu liczby prób udanych do liczby wszystkich. Pojedyncza próba będzie składać się z:
\begin{enumerate}
\itemsep0em 
\item Utworzenia grafu
\item Testowania połączeń
\item Sprawdzenia spójności
\end{enumerate}
Testowanie połączeń będzie polegać na wylosowaniu pesudolosowej liczby z zakresu $[0; 1]$
i porównaniu tej liczby z podaną niezawodnością - jeśli liczba przekroczy niezawodność oznacza to, że dane połączenie zostanie przerwane.

Po przetestowaniu połączeń, sprawdzam czy istnieje połączenie pomiędzy dowolnymi wierzchołkami - jeśli nie, oznacza to że graf jest niespójny, a sieć oparta na takim grafie zawodna. 

Niezawodność sieci obliczam ze wzoru $P = \frac{A}{\Omega}$, gdzie $A$ oznacza liczbę prób zakończonych powodzeniem (sieć spójna), a $\Omega$ liczbę wszystkich prób.

\newpage
\subsection{Wyniki}
\subsubsection{Test nr 1}
W teście nr 1 korzystamy z modelu podstawowego, podanego we wstępie. Po wykonaniu symulacji uzyskujemy następujące wyniki:
\begin{table}[h!]
	\centering
    \label{tab:table1}
    \begin{tabular}{|c|c|c|c|}
    		\multicolumn{4}{c}{Liczba prób}\\
    		\hline
      	& 1,000000 & 10,000,000 & 100,000,000\\
      	\hline
      	Niezawodność & 0.377773 & 0.3775255 & 0.37736323\\
		\hline
    \end{tabular}
\end{table}

\noindent Model sieci:
\begin{figure}[h!]
	\centering
	\includegraphics[width=0.8\linewidth]{graph.jpg}
\end{figure}

\newpage
\subsubsection{Test nr 2}
W teście nr 2 korzystamy z modelu podstawowego z dodatkowym połączeniem pomiędzy wierzchołkami 1 i 20 o niezawodności wynoszteochęącej 0.95. Po wykonaniu symulacji uzyskujemy następujące wyniki:
\begin{table}[h!]
	\centering
    \label{tab:table2}
    \begin{tabular}{|c|c|c|c|}
    		\multicolumn{4}{c}{Liczba prób}\\
    		\hline
      	& 1,000,000 & 10,000,000 & 100,000,000\\
      	\hline
      	Niezawodność & 0.735969 & 0.735715 & 0.73583365\\
		\hline
    \end{tabular}
\end{table}

\noindent Model sieci:
\begin{figure}[h!]
	\centering
	\includegraphics[width=0.8\linewidth]{graph2.jpg}
\end{figure}

\newpage
\subsubsection{Test nr 3}
W teście nr 3 korzystamy z modelu z poprzedniego test z dodatkowymi połączeniami pomiędzy wierzchołkami 1 i 10 oraz 5 i 15 o niezawodnościach wynoszących odpowiednio 0.8 i 0.7. Po wykonaniu symulacji uzyskujemy następujące wyniki:
\begin{table}[h!]
	\centering
    \label{tab:table3}
    \begin{tabular}{|c|c|c|c|}
    		\multicolumn{4}{c}{Liczba prób}\\
    		\hline
      	& 1,000,000 & 10,000,000 & 100,000,000\\
      	\hline
      	Niezawodność & 0.870627 & 0.8706126 & 0.87050279\\
		\hline
    \end{tabular}
\end{table}

\noindent Model sieci:
\begin{figure}[h!]
	\centering
	\includegraphics[width=0.8\linewidth]{graph3.jpg}
\end{figure}

\newpage
\subsubsection{Test nr 4}
W teście nr 4 korzystamy z modelu z poprzedniego testu z dodatkowymi czterema połączeniami, o niezawodności wynoszącej $0.4$, pomiędzy losowymi wierzchołkami. Po wykonaniu symulacji uzyskujemy następujące wyniki:
\begin{table}[h!]
	\centering
    \label{tab:table4}
    \begin{tabular}{|c|c|c|c|}
    		\multicolumn{4}{c}{Liczba prób}\\
    		\hline
      	& 1,000,000 & 10,000,000 & 100,000,000\\
      	\hline
      	Niezawodność & 0.902696 & 0.9030207 & 0.90292538\\
		\hline
    \end{tabular}
\end{table}

\noindent Przykładowy model sieci:
\begin{figure}[h!]
	\centering
	\includegraphics[width=0.8\linewidth]{graph4.jpg}
\end{figure}

\newpage
\subsection{Omówienie wyników}
Dla bardziej skomplikowanych sieci problemem jest analityczne policzenie niezawodności. Jednak dla sieci z testu nr 1 można obliczyć ją w prosty sposób. Sieć pozostaje spójna jeśli żaden z kanałów komunikacyjnych nie jest przerwany, czego prawdopodobieństwo wynosi $P = 0,95^{19} \approx 0.37735360$.

Obliczone prawdopodobieństwo jest bardzo zbliżone do uzyskanego wyniku, a na dokładność wpływa między innymi jakość użytego generatora pseudolosowego.

Jak można zauważyć, w teście nr 2, po dodaniu kanału pomiędzy wierzchołkami $1$ i $20$ prawdopodobieństwo znacząco wzrasta - wynika to z faktu, iż w modelu drugim po zniszczeniu jednego kanału komunikacyjnego sieć wciąż zachowuje spójność.

Podobnie w teście nr 3 i nr 4 dodawanie krawędzi zwiększa niezawodność sieci. W teście nr 4 na niezawodność sieci wpływa również to, które wierzchołki łączą wylosowane krawędzie - im większa odległość pomiędzy wylosowanymi wierzchołkami, tym większa niezawodność sieci.

\newpage
\section{Zadanie nr 2}
\subsection{Obliczanie średniego opóźnienia}
W celu testowania sieci posłużyłem się programem napisanym w Javie przy użyciu biblioteki JGraphT. Program na wejście przyjmuje niezbędne dane takie jak: połączenia pomiędzy wierzchołkami, wielkość pakietu, oraz macierz natężeń strumienia pakietów. Program analizuje spójność sieci, oblicza funkcję przepływu oraz średnie opóźnienie pakietu ze wzoru:
\begin{equation*}
	T = \frac{1}{G} * \sum_e{\frac{a(e)}{\frac{c(e)}{m}-a(e)}}
\end{equation*}
W celu obliczenia funkcji przepływu, program szuka najkrótszej ścieżki pomiędzy wierzchołkami za pomocą algorytmu Dijkstry, a następnie liczy ilość pakietów przechodzących przez dany wierzchołek.


\subsection{Przykładowy model sieci}
W zadaniu posłużyłem się modelem sieci o topologii siatki zgodnie z poniższym schematem:
\begin{figure}[h!]
	\centering
	\includegraphics[width=0.8\linewidth]{graph5.png}
\end{figure}

Wartość przesyłanego pakietu ustawiłem na $10B$.
Dla podanego modelu sieci początkowo ustawiłem poniższą macierz natężeń (liczby pakietów):
\begin{equation*}
	N = [n(i,j)], \forall i,j; i \neq j: n(i,j)=1000
\end{equation*}

Przepustowość na kanałach komunikacyjnych wynosi:
\begin{equation*}
	\forall_{e\in E}: c(e) = 200000\frac{B}{s}
\end{equation*}

W celu policzenia funkcji przepływu, czyli faktycznej liczby pakietów wprowadzanej do poszczególnych kanałów komunikacyjnych posłużyłem się wcześniej wspomnianym programem. Otrzymałem następujący wynik:
\iffalse
\begin{align*}
	a(1, 2) &= 13000 \\
	a(2, 3) &= 15000 \\
	a(3, 4) &= 14000 \\
	a(4, 5) &= 13000 \\
	a(5, 6) &= 13000 \\
	a(6, 7) &= 13000 \\
	a(7, 8) &= 13000 \\
	a(8, 9) &= 12000 \\
	a(9, 10) &= 11000 \\
	a(10, 1) &= 11000 \\
	a(1, 6) &= 8000 \\
	a(2, 7) &= 8000 \\
	a(3, 8) &= 9000 \\
	a(4, 9) &= 7000 \\
	a(5, 10) &= 8000 
\end{align*}
\fi
\begin{figure}[h!]
	\centering
	\includegraphics[width=0.8\linewidth]{graph6.png}
\end{figure}

Jak można zauważyć, ilość przesyłanych pakietów na żadnym kanale nie przekroczyła przepustowości (200000B = 20000 pakietów).

Średnie opóźnienie pakietu w podanym modelu sieci wyniosło:
\begin{equation*}
	T = 0.000023 s
\end{equation*}

\newpage
W przypadku próby modyfikowania sieci tak, aby łączny przepływ na wierzchołkach zbliżał się do przepustowości, opóźnienie sieci znacząco wzrasta. Przykładowo, przy następujących wartościach łącznego przepływu:

\begin{figure}[h!]
	\centering
	\includegraphics[width=0.8\linewidth]{graph7.png}
\end{figure}

średnie opóźnienie pakietu znacząco wzrasta do wartości $T = 0.833 s$

\subsection{Testowanie sieci}
W celu testowania sieci posłużyłem się poprzednim programem rozbudowanym dodatkowo o testowanie nieuszkodzenia krawędzi. Program przyjmuje dodatkowo parametry $p$ (prawdopodobieństwo nieuszkodzenia krawędzi) oraz tmax (maksymalne średnie opóźnienie pakietu). Po utworzeniu sieci program testuje spójność sieci, za pomocą sposobu opisanego w pierwszym zadaniu. Jeśli sieć pozostaje spójna, program testuje wysyłanie pakietów, oblicza całkowity przepływ i średnie opóźnienie. Pojedynczy test kończy się sukcesem, jeśli sieć pozostaje spójna, a średnie opóznienie jest mniejsze niż $tmax$.

\newpage
\subsection{Przykładowy test}
W calu testowania przykładowej sieci posłużę się modelem sieci z zwiększoną liczbą pakietów z poprzedniego zadania. Wartość średniego opóżnienia dla podanego modelu wyniosła $T = 0,769 s$. Doświadczenia wykonuję dla 100000 testów. Wyniki doświadczenia umieszczam w poniższej tabeli:

\begin{table}[h!]
	\centering
    \label{tab:table5}
    \begin{tabular}{|c|c|c|c|}
    		\hline
      	P & 1 & 0.95 & 0.50\\
      	\hline
      	TMax & 0.77 & 0.77 & 0.77\\
      	\hline
      	Niezawodność & 1 & 0.1439 & 0.0002\\
		\hline
    \end{tabular}
\end{table}

W powyższym modelu łączny przepływ był bardzo zblizony do przpustowości sieci. Oznacza to, że w przypadku przerwania chociaż jednego węzła komunikacyjnego istnieje bardzo duże prawdopodbieństwo, że sieć nie będzie potrafiła obsłużyć użytkowników, co można zauważyć w powyższej tabeli - wraz z zmniejszaniem prawdopobdobieństwa nieuszkodzenia kanłów komunikacyjnych niezawodność sieci znacząco się zmniejsza.

\end{document}