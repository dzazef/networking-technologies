\documentclass[titlepage]{article}
\usepackage[utf8]{inputenc}
\usepackage{polski}
\usepackage{amsmath}
\usepackage{listings}
\title{Sprawozdanie 3}
\date{2019-04-20}
\author{Józef Piechaczek}

\begin{document}

\pagenumbering{gobble}
\maketitle
\newpage
\pagenumbering{arabic}

\section{Zadanie 1}
Celem zadania 1 jest napisanie programu ramkującego zgodnie z zasadą "rozpychania bitów" oraz weryfikującego poprawność ramki metodą CRC. Program pobiera z pliku testowego ciąg złożony ze znaków '0' i '1', który symuluje strumień bitów i tworzy na jego podstawie ramki, które następnie zapisuje do pliku wyjściowego. Nazwy plików wejściowych i wyjściowych przyjmowane są jako argumenty wywołania programu.

Algorytm zamieniania strumienia bitów na ramki:
\begin{enumerate}
\itemsep0em
\item Pobranie fragmentu strumienia bitów
\item Obliczenie sumy kontrolnej CRC
\item Konkatenacja fragmentu i sumy CRC
\item Dodanie wartości "0" po każdym ciągu "011111"
\item Dodanie sekwencji "01111110" na początku i końcu ramki
\end{enumerate}

Program obsługuje również procedurę odwrotną, czyli zamianę ramek na tekst wyjściowy.

Algorytm zamieniania ramek na strumień bitów:
\begin{enumerate}
\itemsep0em
\item Pobranie kolejnej ramki
\item Odnalezienie początku i końca wiadomości na podstawie ułożenia sekwencji "01111110"
\item Usunięcie wartości "0" po każdym ciągu "011111"
\item Oddzielenie pola CRC od wiadomości
\item Obliczenie sumy CRC dla uzyskanej wiadomości i porównanie z podanym CRC
\item Zwrócenie wiadomości, jeśli sumy CRC się zgadzają, lub błędu, dla różnych sum kontrolnych
\end{enumerate}

\subsection{Przykład 1}
Umieśćmy w pliku wejściowym następujący ciąg bitów:
\begin{equation*}
011111101010000111110111
\end{equation*}
Po zamienieniu wiadomości na ramki uzyskujemy następujący tekst
\begin{equation*}
01111110\;011111\textbf{0}10101000011111\textbf{0}0111\;10001110100010100000100110100101\;01111110
\end{equation*}
gdzie 
\begin{itemize}
\itemsep0em
\item pierwsza i ostatnia część to sekwencje oddzielające
\item druga część to wiadomość z zaznaczonymi dodatkowymi bitami
\item trzecia część to kod CRC
\end{itemize}
Powyższy tekst zostaje umieszczony w pliku wynikowym.

W celu przetestowanie poprawności ramkowania posłużę się metodą odwrotną - zamieniającą ramki z poprzedniego pliku wyjściowego na źródłowy strumień bitów i zapisującą do poprzedniego pliku wejściowego. Wartość po wykonaniu operacji odwrotnej jest identyczna do wartości sprzed ramkowania.

\subsection{Przykład 2}
W przykładzie drugim posłużę się ciągiem bitów z poprzedniego podpunktu. Uruchamiam program tworzący ramki na postawie strumienia bitów. W pliku wyjściowym umieszczona zostaje następująca ramka z wiadomością:
\begin{equation*}
01111110\;01111101010100001111100111\;10001110100010100000100110100101\;01111110
\end{equation*}
Zamieniam kilka bitów w części zawierającej sumy kontrolne CRC:
\begin{equation*}
01111110\;01111101010100001111100111\;100011101000101\textbf{11111}100110100101\;01111110
\end{equation*}
Podczas uruchamiania programu uzyskuję następujący rezultat:
\begin{lstlisting}
Exception in thread "main" exceptions.IncorrectCRCException
	at crc.Decoder.fromFrame(Decoder.kt:19)
	at crc.Decoder.decodeMessage(Decoder.kt:25)
	at Zad1Kt.decode(Zad1.kt:33)
	at Zad1Kt.main(Zad1.kt:38)
\end{lstlisting}
Wyjątek oznacza że sumy kontrolne nie są równe, co informuje o niezamierzonej zmianie bitów.

\section{Zadanie 2}
Celem zadania 2 jest napisanie symulacji metody dostępu do medium transmisyjnego CSMA/CD. Łącze realizowane jest za pomocą tablicy. Czas działania symulatora podzielony jest na interwały o określonej długości. W danym interwale ramki znajdujące się na łączu zostają przesunięte, a każda ze stacji podejmuje decyzję, jakie działanie powinna podjąć. 

W danym interwale stacje sprawdzają, czy jest możliwość wysyłania ramek - jeśli łącze w danym miejscu jest wolne, stacje z określonym prawdopodobieństwem mogą zacząć transmisję, jeśli łącze jest zajęte, stacje milczą.

Możliwa jest sytuacja, gdy dwie lub więcej stacji zaczną transmisję jednocześnie. Wtedy każda ze stacji, po zauważeniu nieprawidłowych pakietów, wstrzymuje transmisję na okres czasu propagacji jednego pakietu. Po upływie tego czasu każda ze stacji losuje szczeliny czasowe zgodnie ze wzorem 
\newpage
\[   
T = 
     \begin{cases}
       t * R(2^k) &\quad\text{, gdy }1 \leq k \leq 10\\
       t * R(2^{10}) &\quad\text{, gdy }10 \leq k \leq 16\\
     \end{cases}
\]
gdzie
\begin{itemize}
\itemsep0em
\item $R(x)$ - losowa liczba naturalna z zakresu $[0, x)$
\item $k$ - numer kolizji
\item $t$ - czas propagacji pakietu
\end{itemize}







\end{document}